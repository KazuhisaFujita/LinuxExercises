%%%%%%%%%%%%%%%%%%%%%%%%%%%%%
%Since : Jan/29/2009
%Update: <Mar/21/2012>
% -*- coding: utf-8 -*-
%%%%%%%%%%%%%%%%%%%%%%%%%%%%%

\chapter{ネットワーク基礎演習4}

\section{目的}

ネットワークは常に安定に動作しているとは限らない。突然サーバが動かなくなる、つな
がらなくなるなど原因不明の障害が発生する。また、サーバソフトがフリーズするなど、ソフト
が正常に動作しない場合もある。そのようなトラブルに対処するために、プロセスの制御
やネットワークの状態を知るコマンドを知っておくことが重要である。本実験では、プロ
セス制御とネットワークの状況を調査するコマンドを学ぶことを目的とする。

\section{fore groundとback ground}

UNIXはマルチタスクに対応したOSであるため、複数のプログラムを切り替えながら並行に処理を進めることが
できる。私たちがMicrosoft Windowsを利用するときに、WordとExcelを同時に起動しておけるのも、Windowsが
マルチタスクに対応しているからである。複数のタスクのうち、ユーザからの入力を受け取れるものを「フォアグラウンドのタスク」と呼び、その他のものを「バックグラウンドのタスク」という。WindowsやUNIXのX-Windowのようなマルチウインドウ・システムでは、最も手前にあるウインドウがフォアグラウンドに設定される。

\paragraph{演習}
\begin{enumerate}
\item コマンドプロンプトに続いて、``gedit''と入力してEnterキーを押す。
\item テキストエディタのウインドウが表示されると、コマンドプロンプトが表示されないことを確認する。この
とき、``gedit''がフォアグラウンドになって、端末エミュレータがバックグラウンドになっている。
\item テキストエディタのウインドウの右上にある×印のアイコンをクリックして、エディタを終了する。
\item 端末エミュレータがフォアグラウンドになり、コマンドプロンプトが表示されることを確認する。
\end{enumerate}

以上の操作から分かるように、連続して動作したままになるプログラムを起動すると、そのプログラムが終了
するまで端末エミュレータが使用できなくなる。そこで、UNIXではプログラムをバックグラウンドで起動するた
めの方法が用意されている。

\paragraph{演習}
\begin{enumerate}
\item コマンドプロンプトに続いて、``gedit \&''と入力してEnterキーを押す。
\item テキストエディタが起動した後に、端末エミュレータにコマンドプロンプトが表示されていることを確認する。
\end{enumerate}

以上の操作で分かるように、コマンドの後に``\&''を付与すると、そのコマンドがバックグラウンドで起動さ
れる。

Microsoft Windowsでも、UNIXのX-Window(私たちが見ているマルチウインドウ・システム)でも、アイコン
やメニューをクリックすることにより、複数のプログラムを起動することができる。これは、OSの内部では、ア
イコンやメニューをクリックする動作をプログラムのバックグラウンドでの実行に対応付けているからである。
つまり、マウスに操作によって、先ほど学習したようなテキストの命令を生成している。

\subsection{ps}

ターミナルを一見すると、バックグラウンドでプログラムが動いているようには見えない。
しかし、実際は様々なプログラムがバックグラウンドで動作している。Linuxでは、実行
しているプログラムのことをプロセスと呼び、それぞれに番号(プロセスID)を割り振り管
理している。そのプロセスの状況を見るコマンドがpsである。

\paragraph{演習}
\begin{enumerate}
\item ``gedit\&''と入力する。
\item ``ps''と入力し、``gedit''のプロセスがあることを確認する。
\item ``gedit''を閉じる。
\item ``ps''と入力し、``gedit''のプロセスがなくなったことを確認する。
\item ``ps -A''と入力し、現在起動中の全てのプロセスを確認する。
\end{enumerate}

\subsection{top}
CPUやメモリの使用状況などプロセスの状況をリアルタイムで見るコマンドが、topである。

\begin{enumerate}
\item ``top''と入力してEnterキーを押す。現在実行中のプロセスが表示され、CPU
とメモリの使用状況がリアルタイムで更新される。
\item ``q''を入力するとtopが終了し、プロンプトの画面に戻る。
\end{enumerate}

\subsection{kill}

自作のプログラムに限らず、ウェブサーバのようなプログラムもフリーズする可能性があ
る。特に、バックグラウンドで実行されているプログラムが暴走した場合、GUIのソフト
のように閉じるボタンを押すと強制終了の画面が出てきて終了できるように簡単には終了
できない。そのような場合、``top''でプログラムを探し、``ps -A''で暴走しているプロ
グラムのプロセスの番号を確認したあと、``kill''コマンドで終了させる。

\begin{enumerate}
\item ``gedit\&''を入力し実行する。
\item ``ps''と入力し、``gedit''のプロセスを確認する。
\item ``kill プロセスID''もしくは``kill -kill プロセスID''を入力し実行する。実
      行すると``gedit''が閉じられることを確認する。
\end{enumerate}

\section{ネットワークコマンド}

Linuxではネットワークの状況を調査するコマンドが標準で豊富に搭載されている。今回
は、ネットワークの接続やIPアドレス、ホスト名などを調査するコマンドを学ぶ。

\subsection{ping}

ICMPパケットを送り、その応答が帰ってくるかどうかを調べるコマンドである。
主に、コンピュータがネットワークにつながっているかどうかを調べるのに使われる。

\paragraph{演習}
\begin{enumerate}
\item ネットワークに接続しているウェブサーバcpjwebにpingを打つ。pingを使うには
      ターミナルに、``ping cpjweb.center.tsuyama-ct.ac.jp''を入力し実行する。
\item pingをネットワークに接続しているcpjwebに打つとどのような応答するか記録する。
\item ネットワークに接続していないcpjwebにpingを打つ。
\item pingをネットワークに接続していないcpjwebに打つとどのように応答するか記録する。
\end{enumerate}

\subsection{who/whoami}
現在使用しているユーザの名前やログインした時刻を調べる。

\paragraph{演習}
\begin{enumerate}
\item ターミナルに、``who''と入力し実行する。
\item 出力を記録する。
\end{enumerate}

\subsection{hostname}

``hostname''はコンピュータに設定したホスト名(コンピュータの名前)を表示するコマン
ドである。

\paragraph{演習}
\begin{enumerate}
\item ターミナルに、``hostname''と入力してEnterキーを押す。
\item 実行すると、使用しているコンピュータのホスト名が表示される。表示されたホス
      ト名を記録する。
\end{enumerate}

\subsection{ifconfig}

``ifconfig''はコンピュータに設定したIPアドレスやMACアドレスなどのネットワーク
の設定に関する情報を表示するコマンドです。ここでは、自分の使っているコンピュータ
のIPアドレスを調べる。なお、ifconfigの表示はOSにより変わるので注意すること。

\paragraph{演習}
\begin{enumerate}
\item ターミナルに、``/sbin/ifconfig''と入力してEnterキーを押す。
\item 実行すると、様々な情報が表示される。その中でIPアドレスはinetアドレスの後に
      表示され、MACアドレスはハードウェアアドレスに表示される。表示されたIPアド
      レス、MACアドレスを記録せよ。
\end{enumerate}

\subsection{netstat}
コンピュータの外部とのネットワーク通信状態やソケットやインターフェースごとのネッ
トワークの状態を調べるコマンドである。このコマンドを用いることで、コンピュータが
どのような相手と今つながっているのか、どのような通信を行っているのかがわかる。
表示結果の外部アドレスと書かれている部分がどこのコンピュータからアクセスされてい
るかを示す。

\paragraph{演習}
\begin{enumerate}
\item ターミナルに、``netstat -t -u''と入力し実行する。
\item 実行すると、ネットワークの状況(どこのコンピュータとどう通信しているか)が表示さ
      れる。表示結果を記録せよ。
\end{enumerate}


\subsection{nslookup}
ネームサーバに対して、正引きと逆引きの確認をするコマンドである。今回はcpjwebのIPアド
レスを調べ、調べたIPアドレスを使って、ホームページにアクセスしてみる。また、調べ
たIPアドレスからnslookupを使い、ホスト名を調べる。

\paragraph{演習}
\begin{enumerate}
\item ターミナルに、``nslookup cpjweb.center.tsuyama-ct.ac.jp''を打ち込み実行する。
\item 実行すると、IPアドレスが表示される。表示されたIPアドレスを記録する。
\item ブラウザを開き、調べたIPアドレスを使いホームページを開く。URLは
      ``http://IPアドレス/\~{}c-○○○/''である。
\item IPアドレスで指定しても、実験で作成したホームページが表示されることを確認す
      る。
\item IPアドレスからホスト名を調べるため、ターミナルに``nslookup IPアドレス''と
      打ち込み実行する。出力結果を記録し、cpjweb.center.tsuyama-ct.ac.jpであった
      か確認する。
\end{enumerate}

\subsection{arp}
ARPテーブルを表示する。ARPテーブルとは、IPアドレスとMACアドレスの対応の一覧
のことを示します。

\paragraph{演習}
\begin{enumerate}
\item ターミナルに、``arp -a''と入力してEnterキーを押す。
\item 出力結果を記録しなさい。
\end{enumerate}

\section{演習課題}

\begin{enumerate}
\item ``ps -A''を実行し、実行されているプログラムの名前とそのプロセスID
を3つかけ。

\item pingの結果がウェブサーバがネットワークにつながっている状態とつながっていな
      い状態ではどのような結果になったか書け。

\item 演習で自分が使ったPCのホスト名を書け。

\item netstatの表示結果を書け。

\item 自分の使ったPCのIPアドレス、MACアドレスを書け。

\item ウェブサーバcpjwebのIPアドレスは何番か書け。

\item ウェブページを表示する際、ホスト名ドメイン名(cpjweb.center.tsuyama-ct.ac.jp)
      でも、IPアドレスどちらを指定しても表示できた。表示できる理由を簡単に説明せ
      よ。

\item 172.20.20.102のホスト名は何か書け。

\item ``arp -a''を実行した結果を書け。

\item ファイルサーバが突然アクセスしなくなったとする。調査のためpingコマンドでファ
      イルサーバの状態を確認したとき、pingの応答が帰ってきた場合と帰ってこなかった場合
      について、ファイルサーバがどのような状態か推測せよ。(ファイルサーバはソフ
      トそれともハード?)

\item コンピュータのファンの音が通常時よりも大きくなりファンの回転数が上がっている。どう
      もCPUの温度が通常よりも高温になっているようだ。このまま放置しておくとコンピュータ
      が故障する可能性がある。今日習ったコマンドを用い、どのように対処すればよい
      か考えよ。(CPUの温度が高くなるのはどんな時?)
\end{enumerate}
